\section{Elektrický valčík}
\footnotesize\textbf{Cmi, G\7, Cmi, G\7}\\
\\
\normalsize
\ch{Cmi}Jednoho letního večera na návsi pod starou \ch{G\7}lípou\\
Hostinský Antonín Kučera vyvalil soudeček \ch{Cmi}s pípou\\
Neby\ch{A\be}lo to posvícení, neby\ch{Cmi}la to neděle\\
V naší \ch{A\be}obci mezi kopci plni\ch{G\7}ly se korbele\\
\\
\ch{C}Byl to ten slavný den, kdy k nám byl zaveden \ch{G}elektrický proud\\
\ch{G\7}Byl to ten slavný den, kdy k nám byl zaveden \ch{C}elektrický proud\\
\ch{C\7}Střída\ch{F}vý, \ch{G}střída\ch{Emi}vý,  \ch{Ami}silný \ch{Dmi}elek\ch{G\7}trický \ch{C}proud\\
\ch{C\7}Střída\ch{F}vý, \ch{G}střída\ch{Emi}vý,  \ch{Ami}zkrátka \ch{Dmi}elek\ch{G\7}trický proud\ch{Cmi, G\7 , Cmi, G\7}\\
\\
A nyní, kdo tu všechno byl:\\
Okresní a krajský inspektor, hasičský a recitační sbor\\
Poblíže obecní váhy tříčlenná delegace z Prahy,\\
Zástupci nedaleké posádky pod velením poručíka Vosátky,\\
Početná skupina montérů, jeden z nich pomýšlel na dceru sedláka Krušiny\\
Dále krojované družiny, alegorické vozy, italský zmrzlinář Amadeo Cosi\\
Na motocyklu Indián a svatý Jan, z kamene vytesán\\
\\
Byl to ten slavný den...\\
\\
Z projevu inženýra Maliny, zástupce Elektrických podniků:\\
Vážení občané, vzácní hosté, s elektřinou je to prosté\\
Od pantáty vedou dráty do žárovky nade vraty.\\
Dále proud se přelévá do stodoly, do chléva.\\
Při krátkém spojení dvou drátů dochází k takzvanému zkratu.\\
Kdo má pojistky námi předepsané, tomu se při zkratu nic nestane.\\
Kdo si tam nastrká hřebíky, vyhoří a začne od píky\\
Do každé rodiny elektrické hodiny!\\
\\
Byl to ten slavný den...\\
\\
Na stránkách obecní kroniky ozdobným písmem je psáno:\\
Tento den pro zdejší rolníky znamenal po noci ráno\\
Budeme žít jako v Praze, všude samé vedení\\
Jedna fáze, druhá fáze, třetí pěkně vedle ní\\
\\
Byl to ten slavný den...\\