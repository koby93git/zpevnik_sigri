\section{Nagasaki Hirošima + Mladičká básnířka}
\footnotesize\textbf{G, D, C, D}\\
\\
\normalsize
\ch{G}Tramvají \ch{D}dvojkou \ch{C}jezdíval jsem \ch{D}do Židenic,\ch{G D C D}\\
\ch{G}z tak velký \ch{D}lásky \ch{C}většinou \ch{D}nezbyde \ch{Emi}nic,\\
\ch{C}z takový \ch{G}lásky \ch{C}jsou kruhy \ch{G}pod oči\ch{D}ma\\
\ch{G}a dvě spálený \ch{D}srdce – \ch{C}Nagasaki, \ch{D}Hirošima.\ch{G D C D}\\
\\
Jsou jistý věci, co bych tesal do kamene,\\
tam, kde je láska, tam je všechno dovolené,\\
a tam, kde není, tam mě to nezajímá,\\
jó, dvě spálený srdce – Nagasaki, Hirošima.\\
\\
Já nejsem svatej, ani ty nejsi svatá,\\
ale jablka z ráje bejvala jedovatá,\\
jenže hezky jsi hřála, když mi někdy bylo zima,\\
jó, dvě spálený srdce – Nagasaki, Hirošima.\\
\noindent\rule{\textwidth}{1pt}\\
\\
\begin{multicols}{2}
Mladičká básnířka s korálky nad kotníky\\
bouchala na dvířka paláce poetiky\\
s někým se vyspala, někomu nedala\\
láska jako hobby\\
pak o tom napsala sonet na čtyři doby\\
\\
Své srdce skloňovala podle vzoru Ferlinghetti\\
ve vzduchu nechávala viset vždy jen půlku věty\\
plná tragiky, plná mystiky\\
plná splínu\\
tak jí to otiskli v jednom magazínu\\
\\
Bývala viděna v malém baru U Rozhlasu\\
od sebe kolena a cizí ruka kolem pasu\\
trochu se napila, trochu se opila\\
na účet redaktora\\
za týden nato byla hvězdou mikrofóra\\
\\
Pod paží nosila rozepsané rukopisy\\
ráno se budila vedle záchodové mísy\\
múzou políbená životem potřísněná\\
plná zázraků\\
a pak ji vyhodili z gymlpu i z baráku\\
\columnbreak
\\
Šly řeči okolím že měla něco s Esembáky\\
ať bylo cokoli přestala věřit na zázraky\\
cítila u srdce, jak po ní přešla\\
železná bota\\
tak o tom napsala sonet ze života\\
\\
Pak jednou v pondělí\\
přišla na koncert na koleje\\
a hlasem pokorným prosila o text Darmoděje\\
a jak ho psala tak se dala\\
tichounce do pláče\\
a její slzy kapaly na její mrkváče\\
a její slzy kapaly na její mrkváče\\
a její slzy kapaly na její mrkváče\\
a její slzy kapaly na její mrkváče\\
\end{multicols}
